\documentclass[a4paper]{article}
\usepackage{brent_cv}
\usepackage{pifont}
\renewcommand{\refname}{Publications \hrulefill}
\makeatletter
\newcommand \brentfill {
  \tiny
  \leavevmode \cleaders 
  \hb@xt@ .5em{\hss \textperiodcentered \hss }\hfill \kern \z@
  \normalsize
}
\makeatother
\pagenumbering{gobble}
\newcommand{\isp}{:$\ $} 
\newcommand{\bbull}{\ding{118}}

\begin{document}
\noindent
\begin{minipage}[b][1cm]{.7\textwidth}
  \Huge\textbf{Brent Moran} \\
  \normalsize\emph{Data Engineering \& Analytics}
\end{minipage}
\begin{minipage}[b][1cm]{.4\textwidth}
  \raggedleft
  \ttfamily
  \href{mailto:brentmoran@gmail.com}{brentmoran@gmail.com} \\
  \href{https://github.com/mathemancer}{github.com/mathemancer}
\end{minipage}

\subsection*{Relevant Work Experience \hrulefill}

\begin{itemize}
  \item[\bbull] \textbf{Senior Data Engineer} \emph{Creative Commons},
    Remote \brentfill{} 2019-now\\
    Development and maintainence of data pipelines to support CC Search, a media
    search engine.  Research into impact of organizations using CC licenses.
    Maintainer of CC Catalog project.  Lead internships related to all of the
    above.
  \item[\bbull] \textbf{Full Stack Developer} \emph{Metronom GmbH},
    Berlin \brentfill{} 2018-19\\
    Implementation and maintenance of an internal web app for professional
    users. Development and improvement of pricing algorithms. Implementation of
    data pipeline in GCP.
  \item[\bbull] \textbf{Big Data Engineer \& Analyst} \emph{Haensel AMS
    GmbH}, Berlin \brentfill{} 2017-18\\
    Data analysis, Python development, and development on the AWS cloud.
    Set up and test different algorithms.
\end{itemize}

\subsection*{Education \hrulefill}
\begin{itemize}
  \item[\bbull] \textbf{Master of Science} \emph{Freie
    Universit\"at Berlin} \brentfill{} 2018\\
    Thesis topic:  Polynomial bounds on grid-minor theorem
  \item[\bbull] \textbf{Bachelor of Science, Summa Cum Laude} \emph{University
      of Colorado, Denver} \brentfill{} 2015\\
    Major:  Mathematics \\
    Minor: Economics
  \item[\bbull] \emph{Truman State University}, Kirksville,
    Missouri \brentfill{} 2003-06\\
    Studied music composition and analysis
\end{itemize}

\subsection*{Interesting Projects \hrulefill}
\begin{itemize}
  \item[\bbull] \textbf{Linked Commons Graph Analysis} \brentfill{} 2020-now\\
    Leading an internship to determine impact of CC licenses as well as orgs
    which use them from the
    \href{http://dataviz.creativecommons.engineering/}{Linked Commons} graph
    data set.
  \item[\bbull] \textbf{CC Catalog}
    \href{https://github.com/creativecommons/cccatalog}{(Click for Github repo)}
    \brentfill{} 2019-now\\
    Maintaining the CC Catalog project to gather and index metadata about
    hundreds of millions of images from 3rd party APIs as well as Common Crawl.
    The metadata is then transformed, cleaned, and loaded into a PostgreSQL DB
    for use in CC Search.
  \item[\bbull] \textbf{KVI Recommendations on GCP} \brentfill{} 2019\\
    Extended a data processing job to choose appropriate Key Value Items for
    special competitor-based pricing strategies.  Migrated processing job from
    internal cloud solution to GCP.
  \item[\bbull] \textbf{Dynamic content via AI} \brentfill{} 2018\\
    Designed, implemented, and deployed to production a serverless, AI-driven API
    allowing a client website to provide dynamic content to a user based on
    that user's past behavior.
  \item[\bbull] \textbf{ETL Pipeline on AWS} \brentfill{} 2017\\
    Participated in the design and implementation of a serverless ETL
    (Extract, Transform, and Load) pipeline composed of AWS Lambda
    functions and Athena Queries (started by Lambda functions in most
    cases), controlled and sequenced by a finite state machine (AWS Step
    Function).
  \item[\bbull] \textbf{Social Dynamics Simulation (2015):}
    Designed and implemented a simulation of turnover (churn) present in
    a fictional company, for the purposes of analyzing the effect
    different hiring/promotion/firing policies have on employee outcomes
    in a hierarchical corporate setting.
  \item[\bbull] \textbf{Network influence analysis (2014):}
    Designed and implemented a web crawling program in order to generate
    citation networks from data on MathSciNet. Analyzed these networks in order
    to measure the academic influence of various mathematics papers.
  \item[\bbull] \textbf{Cellular automata (2014):}
    Designed and implemented a simulation of world urbanization consisting of a
    cellular automaton underlying an agent-based simulation.
\end{itemize}

\subsection*{Technical Skills \hrulefill}
\begin{itemize}
  \item[\bbull] \textbf{Programming:}  Python, PySpark, Golang,
    JavaScript, Java (Spring Boot), Scala (Play)
  \item[\bbull] \textbf{Querying:} PostgreSQL, MySQL, AWS Athena, Google
    BigQuery
  \item[\bbull] \textbf{Cloud Providers:} Amazon Web Services, Google
    Cloud Platform
  \item[\bbull] \textbf{Operating Systems:}  Linux, MacOS
  \item[\bbull] \textbf{Other:}  bash, Git, Apache Airflow, Docker, Kubernetes,
    Redis, Jenkins, \LaTeX
\end{itemize}


\subsection*{Conference Talks \hrulefill}
\begin{itemize}
  \item[\bbull] \textbf{Joint Mathematics Meetings:}
    San Antonio, Texas \brentfill{} 2015 \\
    \emph{Ramsey-Minimal Saturation Number for Families of Stars}

  \item[\bbull] \textbf{MAA Mathfest:}
    Portland, Oregon \brentfill{} 2014 \\
    \emph{The 1-Relaxed Modular Edge-sum Labeling Game}

  \item[\bbull] \textbf{PPRUMC:}
    Colorado Springs, Colorado \brentfill{} 2014 \\
    \emph{Ramsey-Minimal Saturation Number for Families of Stars}
\end{itemize}

\begingroup
\renewcommand{\section}[2]{\subsection#1{#2}}%
\nocite{brandt:local,butler:forest,ferrara:ramseyus}
\bibliographystyle{plain}
\bibliography{cvbib.bib}
\endgroup

\subsection*{Other Research Experience \hrulefill} 

\begin{itemize}
  \item[\bbull] \textbf{Willamette Valley REU-RET Consortium for
    Mathematics Research} \\
    \emph{1-Relaxed Modular Edge-sum Labeling Game Number}
    Supervised by Charles Dunn and Jennifer Nordstrom of Linfield
    College, during this REU in competitive graph coloring, we developed
    a new graph labeling scheme based on modular arithmetic, and proved
    a number of results regarding our scheme.
\end{itemize}

\end{document}
